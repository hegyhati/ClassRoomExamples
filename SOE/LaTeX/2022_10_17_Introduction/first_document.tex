\documentclass[twoside, openany]{report}

\usepackage[magyar]{babel}
\usepackage{t1enc}
\usepackage[utf8]{inputenc}

\usepackage[]{amsmath}


\author{Mate \underline{Hegyhati}}
\title{Why Ravenclaw is much better than Slytherin}
\date{\today}


\begin{document}
    \maketitle
    

    \tableofcontents

\chapter{The Boy Who Lived}

\section[About the Dursleys]{About the Dursleys\label{introD}, who are the worst muggles ever lived, but it is not a surprise, as a horcrux lives with them}
Mr and Mrs Dursley, of number four, Privet Drive, were proud to 
say that they were perfectly normal, thank you very much. They 
were the last people you'd expect to be involved in anything 
strange or mysterious, because they just didn't hold with such 
nonsense. 

Mr Dursley was the director of a firm called Grunnings, which 
made drills. He was a big, beefy man with hardly any neck, 
although he did have a very large moustache. Mrs Dursley was 
thin and blonde and had nearly twice the usual amount of neck, 
which came in very useful as she spent so much of her time craning 
over garden fences, spying on the neighbours. The Dursleys had a 
small son called Dudley and in their opinion there was no finer 
boy anywhere. 

The Dursleys had everything they wanted, but they also had a 
secret, and their greatest fear was that somebody would discover 
it. They didn't think they could bear it if anyone found out about 
the Potters. Mrs Potter was Mrs Dursley's sister, but they hadn't 
met for several years; in fact, Mrs Dursley pretended she didn't 
have a sister, because her sister and her good-for-nothing husband 
were as unDursleyish as it was possible to be. The Dursleys 
shuddered to think what the neighbours would say if the Potters 
arrived in the street. The Dursleys knew that the Potters had a 
small son, too, but they had never even seen him. This boy was 
another good reason for keeping the Potters away; they didn't 
want Dudley mixing with a child like that. 

When Mr and Mrs Dursley woke up on the dull, grey Tuesday 
our story starts, there was nothing about the cloudy sky outside to 
suggest that strange and mysterious things would soon be hap¬ 
pening all over the country. Mr Dursley hummed as he picked out 
his most boring tie for work and Mrs Dursley gossiped away 
happily as she wrestled a screaming Dudley into his high chair. 

None of them noticed a large tawny owl flutter past the window. 

At half past eight, Mr Dursley picked up his briefcase, pecked 
Mrs Dursley on the cheek and tried to kiss Dudley goodbye but 
missed, because Dudley was now having a tantrum and throwing 
his cereal at the walls. 'Little tyke,' chortled Mr Dursley as he left 
the house. He got into his car and backed out of number four's 
drive. 


\section{Dumbledores arrival}
It was on the corner of the street that he noticed the first sign 
of something peculiar - a cat reading a map. For a second, Mr 
Dursley didn't realise what he had seen - then he jerked his head 
around to look again. There was a tabby cat standing on the corner 
of Privet Drive, but there wasn't a map in sight. What could 
he have been thinking of? It must have been a trick of the light. 
Mr Dursley blinked and stared at the cat. It stared back. As Mr 
Dursley drove around the corner and up the road, he watched the 
cat in his mirror. It was now reading the sign that said Privet Drive 
- no, looking at the sign; cats couldn't read maps or signs. Mr 
Dursley gave himself a little shake and put the cat out of his 
mind. As he drove towards town he thought of nothing except a 
large order of drills he was hoping to get that day. 

But on the edge of town, drills were driven out of his mind by 
something else. As he sat in the usual morning traffic jam, he 
couldn't help noticing that there seemed to be a lot of strangely 
dressed people about. People in cloaks. Mr Dursley couldn't bear 
people who dressed in funny clothes - the get-ups you saw on 
young people! He supposed this was some stupid new fashion. He 
drummed his fingers on the steering wheel and his eyes fell on a 
huddle of these weirdos standing quite close by. They were whis¬ 
pering excitedly together. Mr Dursley was enraged to see that a 
couple of them weren't young at all; why, that man had to be older 
than he was, and wearing an emerald-green cloak! The nerve of 
him! But then it struck Mr Dursley that this was probably some 
silly stunt - these people were obviously collecting for something 
... yes, that would be it. The traffic moved on, and a few minutes 
later, Mr Dursley arrived in the Grunnings car park, his mind 
back on drills. 

Mr Dursley always sat with his back to the window in his office 
on the ninth floor. If he hadn't, he might have found it harder to 
concentrate on drills that morning. He didn't see the owls 



swooping past in broad daylight, though people down in the 
street did; they pointed and gazed open-mouthed as owl after owl 
sped overhead. Most of them had never seen an owl even at night¬ 
time. Mr Dursley, however, had a perfectly normal, owl-free morn¬ 
ing. He yelled at five different people. He made several important 
telephone calls and shouted a bit more. He was in a very good 
mood until lunch-time, when he thought he'd stretch his legs 
and walk across the road to buy himself a bun from the baker's 
opposite. 

He'd forgotten all about the people in cloaks until he passed a 
group of them next to the baker's. He eyed them angrily as he 
passed. He didn't know why, but they made him uneasy. This lot 
were whispering excitedly, too, and he couldn't see a single 
collecting tin. It was on his way back past them, clutching a large 
doughnut in a bag, that he caught a few words of what they were 
saying. 

'The Potters, that's right, that's what I heard -' 
yes, their son, Harry -' 

Mr Dursley stopped dead. Fear flooded him. He looked back at 
the whisperers as if he wanted to say something to them, but 
thought better of it.\cite{Wong2023} 

He dashed back across the road, hurried up to his office, 
snapped at his secretary not to disturb him, seized his telephone 
and had almost finished dialling his home number when he 
changed his mind. He put the receiver back down and stroked his 
moustache, thinking ... no, he was being stupid. Potter wasn't 
such an unusual name. He was sure there were lots of people 
called Potter who had a son called Harry. Come to think of it, he 
wasn't even sure his nephew was called Harry. He'd never even 
seen the boy. It might have been Harvey. Or Harold. There was no 
point in worrying Mrs Dursley, she always got so upset at any 
mention of her sister. He didn't blame her - if he'd had a sister like 
that... but all the same, those people in cloaks ... 

He found it a lot harder to concentrate on drills that afternoon, 
and when he left the building at five o'clock, he was still so 
worried that he walked straight into someone just outside the door. 

'Sorry,' he grunted, as the tiny old man stumbled and almost 
fell. It was a few seconds before Mr Dursley realised that the man 
was wearing a violet cloak. He didn't seem at all upset at being 
almost knocked to the ground. On the contrary, his face split into 



a wide smile and he said in a squeaky voice that made passers-by 
stare: 'Don't be sorry my dear sir, for nothing could upset me 
today! Rejoice, for You-Know-Who has gone at last! Even 
Muggles like yourself should be celebrating, this happy happy 
day!' 

And the old man hugged Mr Dursley around the middle and 
walked off. 

Mr Dursley (as introduced in \ref{introD} on page \pageref{introD}) stood rooted to the spot. He had been hugged by a 
complete stranger. He also thought he had been called a Muggle, 
whatever that was. He was rattled. He hurried to his car and set 
off home, hoping he was imagining things, which he had never 
hoped before, because he didn't approve of imagination. 

As he pulled into the driveway of number four, the first thing he 
saw - and it didn't improve his mood - was the tabby cat he'd 
spotted that morning. It was now sitting on his garden wall. He was 
sure it was the same one; it had the same markings around its eyes. 

\section*{Some owls I guess}

'Shoo!' said Mr Dursley loudly. 

The cat didn't move. It just gave him a stern look. Was this nor¬ 
mal cat behaviour, Mr Dursley wondered. Trying to pull himself 
together, he let himself into the house. He was still determined 
not to mention anything to his wife. 

Mrs Dursley had had a nice, normal day. She told him over din¬ 
ner all about Mrs Next Door's problems with her daughter and 
how Dudley had learnt a new word ('Shan't!'). Mr Dursley tried to 
act normally. When Dudley had been put to bed, he went into the 
living-room in time to catch the last report on the evening news: 

And finally, bird-watchers everywhere have reported that the 
nation's owls have been behaving very unusually today. Although 
owls normally hunt at night and are hardly ever seen in daylight, 
there have been hundreds of sightings of these birds flying in 
every direction since sunrise. Experts are unable to explain why 
the owls have suddenly changed their sleeping pattern.' The news 
reader allowed himself a grin. 'Most mysterious. And now, over to 
Jim McGuffin with the weather. Going to be any more showers of 
owls tonight, Jim?' 

'Well, Ted,' said the weatherman, 'I don't know about that, but 
it's not only the owls that have been acting oddly today. Viewers as 
far apart as Kent, Yorkshire and Dundee have been phoning in 
to tell me that instead of the rain I promised yesterday, they've 
had a downpour of shooting stars! Perhaps people have been 


celebrating Bonfire Night early - it's not until next week, folks! 
But 1 can promise a wet night tonight.' 

Mr Dursley sat frozen in his armchair. Shooting stars all over 
Britain? Owls flying by daylight? Mysterious people in cloaks all 
over the place? And a whisper, a whisper about the Potters ... 

\section{Tea}
Mrs Dursley came into the living-room carrying two cups of 
tea. It was no good. He'd have to say something to her. He cleared 
his throat nervously. 'Er - Petunia, dear - you haven't heard from 
your sister lately, have you?' 

As he had expected, Mrs Dursley looked shocked and angry. 
After all, they normally pretended she didn't have a sister. 

'No,' she said sharply. 'Why?' 

'Funny stuff on the news,' Mr Dursley mumbled. 'Owls ... 
shooting stars ... and there were a lot of funny-looking people in 
town today ...' 

'So?' snapped Mrs Dursley. 

'Well, I just thought ... maybe ... it was something to do with ... 
you know ... her lot.' 

Mrs Dursley sipped her tea through pursed lips. Mr Dursley 
wondered whether he dared tell her he'd heard the name 'Potter'. 
He decided he didn't dare. Instead he said, as casually as he could, 
'Their son - he'd be about Dudley's age now, wouldn't he?' 

'I suppose so,' said Mrs Dursley stiffly. 

'What's his name again? Howard, isn't it?' 

'Harry. Nasty, common name, if you ask me.' 

'Oh, yes,' said Mr Dursley, his heart sinking horribly. 'Yes, I 
quite agree.' 

He didn't say another word on the subject as they went upstairs 
to bed. While Mrs Dursley was in the bathroom, Mr Dursley crept 
to the bedroom window and peered down into the front garden. 
The cat was still there. It was staring down Privet Drive as though 
it was waiting for something. 

Was he imagining things? Could all this have anything to do 
with the Potters? If it did ... if it got out that they were related to a 
pair of - well, he didn't think he could bear it. 

The Dursleys got into bed. Mrs Dursley fell asleep quickly but 
Mr Dursley lay awake, turning it all over in his mind. His last, 
comforting thought before he fell asleep was that even if the 
Potters were involved, there was no reason for them to come near 
him and Mrs Dursley. The Potters knew very well what he and 



Petunia thought about them and their kind ... He couldn't see how 
he and Petunia could get mixed up in anything that might be 
going on. He yawned and turned over. It couldn't affect them ... 

How very wrong he was. 

Mr Dursley might have been drifting into an uneasy sleep, but 
the cat on the wall outside was showing no sign of sleepiness. It 
was sitting as still as a statue, its eyes fixed unthinkingly on the 
far corner of Privet Drive. It didn't so much as quiver when a car 
door slammed in the next street, nor when two owls swooped 
overhead. In fact, it was nearly midnight before the cat moved at all. 

A man appeared on the corner the cat had been watching, 
appeared so suddenly and silently you'd have thought he'd just 
popped out of the ground. The cat's tail twitched and its eyes 
narrowed. 

Nothing like this man had ever been seen in Privet Drive. He 
was tall, thin and very old, judging by the silver of his hair and 
beard, which were both long enough to tuck into his belt. He was 
wearing long robes, a purple cloak which swept the ground and 
high-heeled, buckled boots. His blue eyes were light, bright and 
sparkling behind half-moon spectacles and his nose was very long 
and crooked, as though it had been broken at least twice. This 
man's name was Albus Dumbledore. 

Albus Dumbledore didn't seem to realise that he had just 
arrived in a street where everything from his name to his boots 
was unwelcome. He was busy rummaging in his cloak, looking for 
something. But he did seem to realise he was being watched, 
because he looked up suddenly at the cat, which was still staring 
at him from the other end of the street. For some reason, the sight 
of the cat seemed to amuse him. He chuckled and muttered, 'I 
should have known.' 


\chapter{Whatever}

He had found what he was looking for in his inside pocket. It 
seemed to be a silver cigarette lighter. He flicked it open, held it 
up in the air and clicked it. The nearest street lamp went out with 
a little pop. He clicked it again - the next lamp flickered into 
darkness. Twelve times he clicked the Put-Outer, until the only 
lights left in the whole street were two tiny pinpricks in the dis¬ 
tance, which were the eyes of the cat watching him. If anyone 
looked out of their window now, even beady-eyed Mrs Dursley, 
they wouldn't be able to see anything that was happening down 
on the pavement. Dumbledore slipped the Put-Outer back inside 




his cloak and set off down the street towards number four, where 
he sat down on the wall next to the cat. He didn't look at it, but 
after a moment he spoke to it. 

'Fancy seeing you here, Professor McGonagall.' 

He turned to smile at the tabby, but it had gone. Instead he was 
smiling at a rather severe-looking woman who was wearing square 
glasses exactly the shape of the markings the cat had had around 
its eyes. She, too, was wearing a cloak, an emerald one. Her black 
hair was drawn into a tight bun. She looked distinctly ruffled. 

'How did you know it was me?' she asked. 

'My dear Professor, I've never seen a cat sit so stiffly.' 

'You'd be stiff if you'd been sitting on a brick wall all day' said 
Professor McGonagall. 

'All day? When you could have been celebrating? I must have 
passed a dozen feasts and parties on my way here.' 

Professor McGonagall sniffed angrily. 

'Oh yes, everyone's celebrating, all right,' she said impatiently. 
'You'd think they'd be a bit more careful, but no - even the 
Muggles have noticed something's going on. It was on their news.' 
She jerked her head back at the Dursleys' dark living-room 
window. 'I heard it. Flocks of owls ... shooting stars ... Well, 
they're not completely stupid. They were bound to notice 
something. Shooting stars down in Kent - I'll bet that was Dedalus 
Diggle. He never had much sense.' 

'You can't blame them,' said Dumbledore gently. 'We've had 
precious little to celebrate for eleven years.' 

'I know that,' said Professor McGonagall irritably. 'But that's no 
reason to lose our heads. People are being downright careless, out 
on the streets in broad daylight, not even dressed in Muggle 
clothes, swapping rumours.' \cite{Wong2023}

She threw a sharp, sideways glance at Dumbledore here, as 
though hoping he was going to tell her something, but he didn't, 
so she went on: 'A fine thing it would be if, on the very day You- 

\chapter{Clever Ravenclaws}

If there are $n$ Slytherins dueling $m$ Ravenclaws, and we know that $n \le m$ then the number of possible pairings is: 

\[
    (m-1) (m-2) \dots (m-n+1) = \prod_{i=1}^{n} (m-i+1)
\]

The same formula in inline text mode: $(m-1) (m-2) \dots (m-n+1) = \prod_{i=1}^{n} (m-i+1)$
% Todo:revise it for m > n case too. 

If we want to have a single Slytherin vs. Slytherin duel, the number of options are:

\[
  \binom{n}{2} = \left(\frac {n(n-1)} {2}\right)
\]

\bibliography{mysources.bib}
\bibliographystyle{apalike}
    
\end{document}